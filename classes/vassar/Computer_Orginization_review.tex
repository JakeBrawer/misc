% Created 2015-12-15 Tue 00:26
\documentclass[11pt]{article}
\usepackage[utf8]{inputenc}
\usepackage[T1]{fontenc}
\usepackage{fixltx2e}
\usepackage{graphicx}
\usepackage{grffile}
\usepackage{longtable}
\usepackage{wrapfig}
\usepackage{rotating}
\usepackage[normalem]{ulem}
\usepackage{amsmath}
\usepackage{textcomp}
\usepackage{amssymb}
\usepackage{capt-of}
\usepackage{hyperref}
\date{\today}
\title{}
\hypersetup{
 pdfauthor={},
 pdftitle={},
 pdfkeywords={},
 pdfsubject={},
 pdfcreator={Emacs 24.5.1 (Org mode 8.3.2)}, 
 pdflang={English}}
\begin{document}

\tableofcontents

\section{Final Review}
\label{sec:orgheadline61}
\subsection{RAM}
\label{sec:orgheadline1}
Is "random" because you can access any byte in any order
\begin{itemize}
\item Baisc unit of storage is a cell
\begin{itemize}
\item 1 bit = 1 cell
\item multiple chips form memory
\end{itemize}
\end{itemize}
\subsection{SRAM (Static RAM)}
\label{sec:orgheadline2}
\begin{itemize}
\item 6 transitors per bit
\item Very stable, no leakage
\item Very fast
\item More expensive
\end{itemize}
\subsection{DRAM (Dynamic RAM)}
\label{sec:orgheadline3}
One transistor per bit
\begin{itemize}
\item can have much larger memory per cost
\item Charge stored on a capicitor, which will leak after a while
\begin{itemize}
\item Therefore the controller has to habitually read entier memory and refresh memory
\end{itemize}
\end{itemize}
\subsection{Nonvolatile Memory}
\label{sec:orgheadline6}
\begin{itemize}
\item SRAM and DRAM are volatile memory
\begin{itemize}
\item Will lose data if powered off
\end{itemize}
\end{itemize}
\subsubsection{Read-only memory (ROM)}
\label{sec:orgheadline4}
Code is hardcoded into the circuitry of the chip.
\begin{itemize}
\item Thus, it is nonvolatile and will never change
\end{itemize}
\subsubsection{Flash memory}
\label{sec:orgheadline5}
\begin{itemize}
\item Electrically erasbale programmable ROM, with partial erase capability.
\begin{itemize}
\item Wears out after 100,000 erasings
\end{itemize}
\end{itemize}
\subsection{Memory Read Transaction}
\label{sec:orgheadline7}
\subsection{Harddisk Geometry}
\label{sec:orgheadline8}
\begin{itemize}
\item Disk consists of platters eachwith two surfaces
\item Each track consists of multiple tracks (concentric circles)
\item Each track is divided into sectors, which are the basic unit of the disk
\end{itemize}
\subsection{Disk access time}
\label{sec:orgheadline9}
\begin{itemize}
\item Average time to access some target sector:
\begin{itemize}
\item T\(_{\text{access}}\) = T\(_{\text{avg}}_{\text{seek}}\) + T\(_{\text{avg}}_{\text{rotation}}\) + T\(_{\text{avg}}_{\text{transfer}}\)
\end{itemize}
\item Seek time for head is biggest limiting factor
\item At 7200 RPM, T\(_{\text{access}}\) = 13.02ms
\item SRAM takes about 4ns to access a 4-bit word
\end{itemize}
\subsection{SSD}
\label{sec:orgheadline10}
\begin{itemize}
\item Organized into pages and blocks
\begin{itemize}
\item Page: 512-4kbs
\item Blocks: 32-128 pages
\item data read/written in units of pages
\end{itemize}
\item Reads much faster tahn HDD
\item Erasing takes long though
\end{itemize}

\subsection{Cache}
\label{sec:orgheadline12}
\subsubsection{Cache size}
\label{sec:orgheadline11}
C = S x E x B
\begin{itemize}
\item C - cache
\item S - number of sets
\item E - number of lines per set
\item B - block size
\end{itemize}
\subsection{Virtual address}
\label{sec:orgheadline15}
\subsubsection{Basic parameters}
\label{sec:orgheadline13}
\begin{itemize}
\item N: 2\(^{\text{n}}\) number of addresses in V
\item M = 2\(^{\text{m}}\) number of addresses in P
\item P = 2\(^{\text{p}}\) page size
\end{itemize}
\subsubsection{Components of a VA}
\label{sec:orgheadline14}
\begin{itemize}
\item TLBT (TLB tag)
\begin{itemize}
\item which page in in the set?
\end{itemize}
\item TLBI (TLB index)
\begin{itemize}
\item which set in the TLB?
\end{itemize}
\item VPO: virual page offset (p bits)
\begin{itemize}
\item index into the page itself (which byte of memory do we want?)
\end{itemize}
\item VPN: Virtual page number  (n-p bits)
\begin{itemize}
\item index into page table
\end{itemize}
\end{itemize}
\subsection{Control Hazards}
\label{sec:orgheadline19}
\subsubsection{Missed Prediction}
\label{sec:orgheadline16}
When a coditional jump is executed is executed prematurally 
\begin{itemize}
\item Fixed by replacing instructions in decode and execute stages with bubbles.
\item instructions effectively cancelled.
\item Waste two cycles
\end{itemize}
\subsubsection{Return}
\label{sec:orgheadline17}
Return address is saved on the stack, so need to wait till MEMORY stage is complete
\begin{itemize}
\item solved by simply placing a few bubbles until address is retrieved.
\item Waste 3 cycles
\end{itemize}
\subsubsection{Load/use}
\label{sec:orgheadline18}
When we retrieve something from memory and the destination of this call is being used as the source in the next call
\subsection{Control conditions}
\label{sec:orgheadline20}
Can have combnations of hazards
\begin{itemize}
\item Combination A
\begin{itemize}
\item jump in the EXECUTE stage and a return in the Decode
\end{itemize}
\begin{itemize}
\item Combination B
\begin{itemize}
\item loading is in EXECUTE stage and a return in the DECODE stage (use)
\end{itemize}
\end{itemize}
\end{itemize}
\subsection{Exceptions}
\label{sec:orgheadline25}
\subsubsection{Causes}
\label{sec:orgheadline21}
\begin{itemize}
\item Halt instruction
\item Bad address for instruction for data
\item Invalid Instruction
\end{itemize}
\subsubsection{Detection}
\label{sec:orgheadline22}
Sometimes an exception will be detected but it shouldnt be triggered
\subsubsection{Maintaning exception ordering}
\label{sec:orgheadline23}
\begin{itemize}
\item Add status field to pipeline register
\end{itemize}
Set by FETCH:
\begin{itemize}
\item AOK: ok
\item ADR: when bad fetch address
Also set by MEMORY
\item HLT: Halt instruction
\item INS: Illegal instruction
\end{itemize}
\subsubsection{Avoiding sideeffects}
\label{sec:orgheadline24}
\begin{itemize}
\item Invalid inst. are converetd into bubbles
\end{itemize}

\subsection{Pipelined Implemention \textit{<2015-11-03 Tue>}}
\label{sec:orgheadline60}
\subsubsection{Basic Idea}
\label{sec:orgheadline26}
\begin{itemize}
\item Divide process into independent stages
\item multiple objects being processed at different stages
\end{itemize}
\subsubsection{Throughput}
\label{sec:orgheadline27}
\begin{itemize}
\item Number of objects processes per unit time
\end{itemize}
\subsubsection{Latency}
\label{sec:orgheadline28}
Time it takes to process a single object

\subsubsection{3-way pipelines version}
\label{sec:orgheadline29}
Break up computation into 3 stages. Save each stage at a register.
\begin{itemize}
\item This allows the first stage, for example, to do more computations more quickly, because it has been freed up by the register
\item Results in much larger throughput, but higher latency
\end{itemize}

\subsubsection{Limitations}
\label{sec:orgheadline32}
\begin{enumerate}
\item Nonuniform Delays
\label{sec:orgheadline30}
\begin{itemize}
\item Throughput limited by slowest stage
\item Other stages then sit idle for a lot of time
\item Difficult to partion system into balanced system
\end{itemize}
\item Register overhead
\label{sec:orgheadline31}
As you try to deepen the pipeline (add more stages), the writing to registers becomes more significant
\begin{itemize}
\item Therefore pipelining is limited by the write speed of the registers.
\end{itemize}
\end{enumerate}
\subsubsection{Data Dependency}
\label{sec:orgheadline33}
If operations in stage A depend on results of stage C, stage A depends on C
\begin{itemize}
\item Therefore pipeline needs to handle this properly, or risk getting the computaton wrong
\end{itemize}
\subsubsection{Pipline architure stuff}
\label{sec:orgheadline36}
\begin{enumerate}
\item S\(_{\text{field}}\)
\label{sec:orgheadline34}
Value of field held in stage S pipeline registers
\item s\(_{\text{field}}\)
\label{sec:orgheadline35}
Value of field computed in stage S
\end{enumerate}
\subsubsection{Feedback path}
\label{sec:orgheadline37}
Anywhere that a particular value goes back to a particular stage
\subsubsection{PC prediciton strategy}
\label{sec:orgheadline40}
\begin{enumerate}
\item Instructions that dont transfer control
\label{sec:orgheadline38}
\begin{itemize}
\item Predict next PC to be valP
\item always reliable
\end{itemize}
\item Call and Unconditional jumps
\label{sec:orgheadline39}
\begin{itemize}
\item Predict next Pc to be valC
\end{itemize}
\end{enumerate}
\subsubsection{Stalling for data dependencies}
\label{sec:orgheadline42}
\begin{enumerate}
\item Bubble
\label{sec:orgheadline41}
If instruction follows too closely after one that writes to a register, slow it down.
\begin{itemize}
\item hold instruction in decode
\item dynamically inject nop into execute stage
\item stalled instruction is held in DECODE stage
\item Sucessive instructions held in the FETCH stage
\end{itemize}
\end{enumerate}
\subsubsection{Implementing Stalling}
\label{sec:orgheadline43}
\subsubsection{Data forwarding}
\label{sec:orgheadline46}
ValE or ValM have been calculated in EXECUTE stage, but have yet to be written to a register 
\begin{itemize}
\item Therefore, therefore the write stage can be bypassed, by forwarding these values to earlier stages if they need them
\item eliminates the need fro nops and bubbles
\end{itemize}
\begin{enumerate}
\item Priority
\label{sec:orgheadline44}
If for some reason there are multiple choices, the earlies stage has priority (EXECUTE has highest priority)
\item Limitation
\label{sec:orgheadline45}
When a register that gets its value from a memory location, and a following instruction needs the value in the registers
\begin{itemize}
\item have to wait to the memory stage to get the value
\item need a bubble
\item 
\end{itemize}
\end{enumerate}

\subsubsection{Control Hazards}
\label{sec:orgheadline50}
\begin{enumerate}
\item Missed Prediction
\label{sec:orgheadline47}
When a coditional jump is executed is executed prematurally 
\begin{itemize}
\item Fixed by replacing instructions in decode and execute stages with bubbles.
\item instructions effectively cancelled.
\item Waste two cycles
\end{itemize}
\item Return
\label{sec:orgheadline48}
Return address is saved on the stack, so need to wait till MEMORY stage is complete
\begin{itemize}
\item solved by simply placing a few bubbles until address is retrieved.
\item Waste 3 cycles
\end{itemize}
\item Load/use
\label{sec:orgheadline49}
When we retrieve something from memory and the destination of this call is being used as the source in the next call
\end{enumerate}
\subsubsection{Control conditions}
\label{sec:orgheadline51}
Can have combnations of hazards
\begin{itemize}
\item Combination A
\begin{itemize}
\item jump in the EXECUTE stage and a return in the Decode
\end{itemize}
\begin{itemize}
\item Combination B
\begin{itemize}
\item loading is in EXECUTE stage and a return in the DECODE stage (use)
\end{itemize}
\end{itemize}
\end{itemize}
\subsubsection{Exceptions}
\label{sec:orgheadline56}
\begin{enumerate}
\item Causes
\label{sec:orgheadline52}
\begin{itemize}
\item Halt instruction
\item Bad address for instruction for data
\item Invalid Instruction
\end{itemize}
\item Detection
\label{sec:orgheadline53}
Sometimes an exception will be detected but it shouldnt be triggered
\item Maintaning exception ordering
\label{sec:orgheadline54}
\begin{itemize}
\item Add status field to pipeline register
\end{itemize}
Set by FETCH:
\begin{itemize}
\item AOK: ok
\item ADR: when bad fetch address
Also set by MEMORY
\item HLT: Halt instruction
\item INS: Illegal instruction
\end{itemize}
\item Avoiding sideeffects
\label{sec:orgheadline55}
\begin{itemize}
\item Invalid inst. are converetd into bubbles
\end{itemize}
\end{enumerate}
\subsubsection{Performance Metrics}
\label{sec:orgheadline59}
\begin{enumerate}
\item Clockrate
\label{sec:orgheadline57}
\begin{itemize}
\item Measured in GhZ
\end{itemize}
\item Rate at which instructions executed
\label{sec:orgheadline58}
\begin{itemize}
\item CPI: cycles per instruction
\begin{itemize}
\item on avergae, how many clock cycles does each instruction recquire?
\end{itemize}
\item in PIPE architecure we want as close to 1 instuct. per clock cycle.
\end{itemize}
\end{enumerate}
\end{document}